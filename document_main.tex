% LaTex template for technical documentations
%
% This template follows the KISS-approach: keep it simple and stupid!
% It is using the KOMA script and is intended to be compiled using pdflatex and biber
% Options are switched by commenting them in or out

%% ----------------------------------------------------------------------------
%% DOCUMENT SETUP

\documentclass[
	a4paper,  	% papersize: a4paper, a5paper, letterpaper
	BCOR=0mm,	% binding correction
	12pt,		% fontsize: 10pt, 11pt, 12pt
	oneside,	% paper usage: oneside, twoside
%	landscape,	% enables the landscape orientation
	onecolumn,	% column setting: onecolumn, twocolumn
%	openany, 	% open chapter position setting (only report and book): openright, openany
]
{scrartcl}	% article
%{scrreprt}	% report
%{scrbook}	% book

%% ----------------------------------------------------------------------------
% PAGE STYLE

\usepackage[
%	headsepline,	% enable header seperation line
%	footsepline,	% enable footer seperation line
]{scrlayer-scrpage}

\clearpairofpagestyles

\ohead{}		% content of right/outer header
\chead{}		% content of center header
\ihead{}		% content of left/inner header

\ofoot{}		% content of right/outer footer
\cfoot{\pagemark}		% content of center footer
\ifoot{}		% content of left/inner footer


%% ----------------------------------------------------------------------------
%% DOCUMENT author & title

\newcommand{\documentorg}{Span d.d.}
\newcommand{\documentauthor}{Ivan Derdić}
\newcommand{\documenttitle}{Cash Management}
\newcommand{\documentsubtitle}{Technical Documentation}
\newcommand{\documentdate}{\today}

%% ----------------------------------------------------------------------------
%% PACKAGES

\usepackage[english]{babel}		% english dictionary
\usepackage[utf8]{inputenc}		% input encoding
\usepackage[T1]{fontenc}		% font encoding in final document

\usepackage{amsmath,amssymb,amstext}	% math symbols and stuff
\usepackage{enumitem}			% small itemize

\usepackage[			% for hyperlinks in the document
	backref=true,		% backwards reference in bibliography
	bookmarks=true,		% create PDF bookmarks
	colorlinks=true,	% generate colored links
	linkcolor=blue,		% color for links
	citecolor=blue,		% color for cites
	filecolor=blue,		% color for local file URLs
	runcolor=blue,		% color for run links
	urlcolor=blue,		% color for URLs
]{hyperref}

\usepackage{eso-pic}			% optional, for absolute placement of titlepace logos
\usepackage[pdftex]{graphicx}	% grpahics package, option set for pdflatex compiler. [dvips] needed for DVI/PS compiler

%% ----------------------------------------------------------------------------
%% MACROS

% generate figure environment
% usage: \placefigure{<reference>}{<path>}{<width_in_fraction_of textwidth>}{<caption>}
\newcommand{\placefigure}[4]{
	\begin{figure}[htp]
		\centering
		\includegraphics[width=#3\textwidth]{figures/#2}
		\caption{#4}
		\label{#1}
	\end{figure}
}

%% ----------------------------------------------------------------------------
%% BIBLIOGRAPHY

\usepackage[backend=biber,	% using biber to compile references
	style=numeric,			% uses numeric indicies
	backref=true, 			% create backward references
	natbib=true, 			%
	hyperref=true, 			% use hyperref package to create links
	sorting=none 			% sort none orders by occurence
]{biblatex}

\addbibresource{references.bib}

%% ----------------------------------------------------------------------------

\begin{document}
\pagenumbering{roman}	% pagenumbering options: roman, Roman, arabic, alph, Alph


%% ----------------------------------------------------------------------------
%% TITLE PAGE

\begin{titlepage}
\begin{center}

\vspace*{5cm}

{\documentorg}

\vspace{.4cm}

{\small\documentauthor}

\vspace{.8cm}

{\LARGE\documenttitle}

\vspace{.2cm}

{\documentsubtitle}

\vfill

{\small\documentdate}

\end{center}
\end{titlepage}
%\begin{titlepage}

	% the following commands places the TUG logo in the upper left corner, requires eso-pic package!
	\AddToShipoutPictureFG*{
		\put(\LenToUnit{ \paperwidth - 30mm },\LenToUnit{ \paperheight - 25mm }){\makebox[0pt][c]{\includegraphics[width=40mm]{titlepages/TU_Graz_logo}}}
	}

	\begin{center}

		\vspace*{5cm}

		{\normalsize\documentauthor}

		\vspace{.8cm}

		{\LARGE\textbf{\documenttitle}}

		{\large\documentsubtitle}

		\vfill

		{\normalsize\documentdate}

	\end{center}
\end{titlepage}


%% ----------------------------------------------------------------------------
%% TABLE OF CONTENT, LISTS OF ...

\tableofcontents
\listoffigures
%\listoftables

\newpage

%% ----------------------------------------------------------------------------
%% MAIN CONTENT

\pagenumbering{arabic}	% pagenumbering options: roman, Roman, arabic, alph, Alph

% document content is inserted here. \input and \include may be used
\section{Introduction} \label{sec:introduction}

Some sectio ntext


%% ----------------------------------------------------------------------------
%% BIBLIOGRAPHY

\newpage
\printbibliography		% print bibliography

%% ----------------------------------------------------------------------------
\end{document}



